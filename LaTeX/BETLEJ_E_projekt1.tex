\documentclass[10pt]{mwart}
\usepackage[utf8]{inputenc}
\usepackage{polski}
\usepackage{amsmath}
\usepackage{amssymb}
\usepackage{amsthm}
\newcommand{\RR}{{\mathbb R}}
\newcommand{\NN}{{\mathbb N}}
\newcommand{\ZZ}{{\mathbb Z}}
\newcommand{\QQ}{{\mathbb Q}}
\newcommand{\CC}{{\mathbb C}}
\usepackage{hyperref}
\usepackage{paralist}
\usepackage[a4paper,top=2cm,bottom=2cm,left=3cm,right=3cm,marginparwidth=1.75cm]{geometry}
\usepackage{xcolor}
\hypersetup{
    colorlinks=true,
    citecolor=violet,
    linkcolor=blue,
    filecolor=magenta,      
    urlcolor=red,}
\usepackage{pifont}
\usepackage{enumerate}
\usepackage{graphicx} 
\theoremstyle{plain} \newtheorem{tw}{Twierdzenie}[section]
\theoremstyle{plain} \newtheorem{lem}[tw]{Lemat}
\theoremstyle{definition} \newtheorem{df}[tw]{Definicja}
\title{Zadania matematyczne}
\author{Emilia Betlej}
\date{4 grudnia 2023}
\begin{document}
\maketitle
\tableofcontents
\section{Analiza matematyczna}
\subsection{Funckje cyklometryczne}
\begin{df}
    Funkcją arccos nazywamy funkcję odwrotną do funkcji kosinus zawężonej do przedziału domkniętego \([0, \pi]\). Funckja arccos wyraże się wzorem: arccos = \((cos_{[0, \pi]})^-1\)\cite{odnośnik3}:
\end{df}
\begin{tw}
   Jeśli a \(\in (0; 1)\) i \(x_1x_2\in \RR_+\) to \(\log_a{x_1} < \log_a{x_2}\iff x_1>x_2\).
\end{tw}
\begin{tw}
   Jeśli a \(\in (1; +\infty)\) i \(x_1x_2\in \RR_+\) to \(\log_a{x_1} < \log_a{x_2}\iff x_1<x_2\)
\end{tw}
\subsection{Tabela wybranych wartości funkcji trygonometrycznych}
\includegraphics[width=0.39\linewidth, angle=0]{tabela.jpg}
\subsection{Zadanie 1}
Rozwiąż nierówność: \(\arccos({log_4{x}-\frac{\pi}{6}})\) \(\geq\) \(\frac{\pi}{6}\)\\
Założenie: \begin{align*}
-1\leq\log_4{x}\leq1 \wedge x > 0\\
\log_4{\frac{1}{4}}\leq\log_4{x}\leq\log_4{4} \wedge x\in(0, +\infty)\\
x\geq\frac{1}{4} \wedge x\leq4 \wedge x\in(0, +\infty)\end{align*}
 \begin{align*}D = [\frac{1}{4}, 4]\end{align*} Obliczenia:
\begin{align}
\arccos(log_4{x})-\frac{\pi}{6}\geq \frac{\pi}{6}\\
\arccos(log_4{x})\geq \frac{\pi}{3}\\
\arccos(log_4{x})\geq arccos(\frac{1}{2})\\
\log_4{x}\leq \frac{1}{2}\\
\log_4{x}\leq \log_4{2}\\
x \leq 2\\
x \in (-\infty, 2] \wedge x\in[\frac{1}{4}, 4]\\
x\in [\frac{1}{4}, 2]
\end{align}
Odpowiedź: Rozwiązaniem nierówności jest \(x\in [\frac{1}{4}, 2]\).
\section{Logika matematyczna}
\subsection{Prawa logiczne}
\begin{enumerate}[a)]
   \item prawa łączności
   \begin{itemize}
       \item p \(\vee\) p \(\equiv\) p
        \item p \(\wedge\) p \(\equiv\) p
       \item p \(\vee\) q \(\equiv\) q \(\vee\) p
       \item p \(\wedge\) q \(\equiv\) q \(\wedge\) p
       \item p \(\vee\) (q \(\vee \) r) \(\equiv\) (p \(\vee \) q) \(\vee\) r
       \item p \(\wedge\) (q \(\wedge\) r) \(\equiv\) (p \(\wedge\) q) \(\wedge\) r
   \end{itemize}
   \item prawo rozdzielności koniungcji względem alternatywy\cite{odnośnik1}
   \begin{itemize}
       \item p \(\wedge\) (q \(\vee\) r) \(\equiv\) (p \(\wedge\) q) \(\vee\) (p \(\wedge\) r)
   \end{itemize}
   \item prawo rozdzielności alternatywy względem koniungci
     \begin{itemize}
       \item p \(\vee\) (q \(\wedge\) r) \(\equiv\) (p \(\vee\) q) \(\wedge\) (p \(\vee\) r)
   \end{itemize}
   \item prawa de Morgana
   \begin{itemize}
       \item \(\neg\) (p \(\wedge\) q) \(\equiv\) \(\neg\) p \(\vee\) \(\neg\) q
       \item \(\neg\) (p \(\vee\) q) \(\equiv\) \(\neg\) p \(\wedge\) \(\neg\) q
   \end{itemize}
   \item prawa absorbcji
   \begin{itemize}
       \item  p \(\vee\) (p \(\wedge\) q) \(\equiv\) p
       \item  p \(\wedge\) (p \(\vee\) q) \(\equiv\) p
   \end{itemize}
   \item prawo podwójnego przeczenia
   \begin{itemize}
       \item \(\neg \neg\) p \(\equiv\) p
   \end{itemize}
   \item prawo eksportacji i importacji
   \begin{itemize}
       \item p \(\implies\) (q \(\implies\) r) \(\equiv\) (p \(\wedge\) q) \(\implies\) r
   \end{itemize}
   \item prawo kontrapozycji
   \begin{itemize}
       \item \(\neg\) p \(\implies\) \(\neg\) q \(\equiv\) g \(\implies\) p
   \end{itemize}
\end{enumerate}
\subsection{Zadanie 2}
Sprawdź czy poniższe zdanie jest tautologią.\\
p \(\implies\) (q \(\implies\) (p \(\wedge\) q))\cite{odnośnik1}\\
Rozwiązanie:\\
Wykorzytam tabelkę, aby sprawdzić czy powyższe zdanie jest tautologią.\\
\(
\begin{array}{|c|c|c|c|c|}
\hline\mbox{p} & q & p \wedge q & q \implies (p \wedge q ) & p \implies ( (p \wedge q ))\\\hline
\mbox{1} & 1 & 1 & 1 & 1 \\\hline
\mbox{0} & 1 & 0 & 0 & 1 \\\hline
\mbox{1} & 0 & 0 & 1 & 1\\\hline
\mbox{0} & 0& 0 & 1& 1 \\\hline
\end{array}\)\\
Zdanie p \(\implies\) (q \(\implies\) (p \(\wedge\) q)) zawsze przyjmuje warość logiczną "1". Wykazałam, że powyższe zdanie jest tautologią.\\
Odpowiedź:\\ Zdanie p \(\implies\) (q \(\implies\) (p \(\wedge\) q)) jest tautologią.
\subsection{Zadanie 3}
Sprawdź, czy poniższy schemat rozumowania jest poprawny.\\
\(\frac{p \implies q, \neg q \implies \neg r}{\neg p \implies (r \vee q)}\)\\
Rozwiązanie:\\ Wykorzystam metodę "nie wprost". Zakładam, że istnieje takie wartościowanie, dla którego przesłanki są prawdziwe, a wniosek jest fałszywy, zatem:\\Wniosek: \(\neg p \implies (r \vee q)\) : 0\\
\(\neg p\) : 1  \((r \vee q)\) : 0, z tego wynika, że p: 0, q: 0, r: 0.\\ Podstawiam wartośći p i r do przesłanek.\\Przesłanki:
\(p \implies q: 1, \neg q \implies \neg r: 1, p: 0, r: 0\)\\
q: 1, r: 0\\ Otrzymałam sprzeczność, zatem schemat rozumowania jest zawsze poprawny.\\Odpowiedź: Schemat rozumowania jest zawsze poprawny.
\section{Algebra liniowa}
\subsection{Macierze}
\begin{df}
    Macieżą nazywamy odwzorowanie {1,2, ..., m} \(\times\) {1,2, ..., n} \(\ni\) (i, j) \(\to a_{ij}\in\) x,\\gdzie m, n \(\in \NN\)\\ X - zbiór (elementami tego zbioru mogą być różne obiekty matematyczne, np. liczby, funkcje itd.)\cite{odnośnik2}
\end{df}
\begin{df}
    Mówimy, że macierz A\(\in M_{m\times n}\) jest w postaci schodkowej, jeśli spełnia następujące warunki:\\
    -każdy wiersz zerowy (składający się z samych zer) znajduje się poniżej każdego wiersza niezerowego\\
    -w każdym wierszu pierwszy (licząc od lewej) niezerowy wyraz znajduje się w kolumnie stojącej na prawo od pierwszego niezerowego wyrazu wiersza poprzedniego.\cite{odnośnik2}
\end{df}

\subsection{Zadanie 4}
Rozwiąż układ równań:
$$\left\{\begin{array}{rcl}
3x - 4y +z&=&9\\
7x + 3y - 5z&=&4\\
2x + 5y + 2z&=&9\\
4x + 3y - z&=&1\\
9x + 2y - 3z&=&13
\end{array} \right.$$\cite{odnośnik2}
Rozwiązanie:\\
\(\begin{bmatrix}
3&-4&1&9\\
7&3&-5&4\\
2&5&2&9\\
4&3&-1&1\\
9&2&-3&13\\
\end{bmatrix}\) \(\to\)
\(\begin{bmatrix}
1&-4&3&9\\
-5&3&7&4\\
2&5&2&9\\
-1&3&4&1\\
-3&2&9&13\\
\end{bmatrix}\) \(\to\)
\(\begin{bmatrix}
1&-4&3&9\\
0&-17&22&49\\
0&3&-4&-9\\
0&-1&7&20\\
0&-14&18&40\\
\end{bmatrix}\) \(\to\)
\(\begin{bmatrix}
1&-4&3&9\\
0&-1&7&20\\
0&3&-4&-9\\
0&-17&22&49\\
0&-7&9&20\\
\end{bmatrix}\) \(\to\)\\ \(\to\)
\(\begin{bmatrix}
1&-4&3&9\\
0&-1&7&20\\
0&0&17&51\\
0&0&-27&291\\
0&0&-40&-120\\
\end{bmatrix}\) \(\to\)
\(\begin{bmatrix}
1&-4&3&9\\
0&-1&7&20\\
0&0&1&3\\
0&0&1&3\\
0&0&1&3\\
\end{bmatrix}\) \(\to\)
\(\begin{bmatrix}
1&-4&3&9\\
0&-1&7&20\\
0&0&1&3
\end{bmatrix}\)
\\ \\ otrzymałam macierz w postaci schodkowej, zatem:
$$\left\{\begin{array}{rcl}
z - 4y +3x&=&9\\
0 - y + 7x&=&20\\
0 + 0 + x&=&3\\
\end{array} \right.$$
$$\left\{\begin{array}{rcl}
x&=&3\\
z - 4y +3x&=&9\\
- y + 21&=&20\\
\end{array} \right.$$
$$\left\{\begin{array}{rcl}
x&=&3\\
y&=&1\\
z - 4 +9&=&9\\
\end{array} \right.$$
$$\left\{\begin{array}{rcl}
x&=&3\\
y&=&1\\
z&=&4\\
\end{array} \right.$$
Odpowiedź: x = 3, y = 4, z = 1.
\begin{thebibliography}{1293}
    \hypertarget{odnośnik1}{\bibitem{odnośnik1} 
Edward Nieznański: "Logika", 2011, Wyd. C.H. Beck.}
     \hypertarget{odnośnik2}{\bibitem{odnośnik2} Jerzy Topp: "Algebra liniowa", 2015, Wydawnictwo Uniwersytetu Gdańskiego.}
     \hypertarget{odnośnik3}{\bibitem{odnośnik3} Włodzimierz Krysicki, Lech Włodarski: "Analiza Matematyczna w Zadaniach. Część 1", 2008, Wydawnictwo Naukowe PWN.}
\end{thebibliography}
\end{document}
